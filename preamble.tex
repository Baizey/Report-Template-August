%%------------%%
%%  PACKAGES  %%
%%------------%%

    \usepackage[english]{babel}                 % English syntax
    \usepackage[titles]{tocloft}                % Nicer table of contents
    \usepackage[export]{adjustbox}              % Framed images
    \usepackage[colorlinks = true, allcolors = s13, linkcolor = black]{hyperref}
                                                % Hyperlinks in the document
    \usepackage{
        natbib,                                 % References
        amsmath, amsfonts, amssymb, everysel,   % Advanced matematic notation
        fancyhdr, lastpage,                     % Header and footer
        lmodern,                                % Nicer standard font
        lipsum,                                 % Sample text
        tcolorbox, colortbl, varwidth,          % Boxes
        array, tabularx, siunitx,               % Tables in said boxes
        fontspec,                               % New fonts
        enumitem,                               % Easier changes in the enumerate environment
        changepage,                             % Change page margin
        sectsty,                                % Change section titles
        xcolor,                                 % More color options
        graphicx,                               % More graphics options
        microtype,                              % Nicer paragraphs
        styles/dtucolors,                       % DTU colours
        listings, minted, mdframed,             % Code listing
        tikz,                                   % Graphs and more
        pgfkeys,                                % Arrays
        forloop,                                % For loops
        etoolbox,                               % If statements
        forest                                  % tableau trees
    }

%%--------------------%%
%%  PACKAGE SETTINGS  %%
%%--------------------%%

	
    % quick styling for tableaus using forest
	
    \forestset{
      tableaux/.style={
        for tree={
          math content
        },
        where n children=1{
          !1.before computing xy={l=\baselineskip},
          !1.no edge
        }{},
        closed/.style={
          label=below:$\times$
        },
        open/.style={
          label=below:$\circ$
        },
        parent anchor=south,
        child anchor=north,
      },
    }

    % No indentation for entire document
        \setlength\parindent{0pt}

    % Table of contents settings
        \renewcommand{\cftsecfont}{\sf}
        \renewcommand{\cftsubsecfont}{\footnotesize\sf}
        \renewcommand{\cftsubsubsecfont}{\scriptsize\sf}
        \renewcommand{\cftsecpagefont}{\sf}
        \renewcommand{\cftsubsecpagefont}{\footnotesize\sf}
        \renewcommand{\cftsubsubsecpagefont}{\scriptsize\sf}
        \renewcommand{\cftsecdotsep}{2}
        \renewcommand{\cftsubsecdotsep}{\cftnodots}
        \renewcommand{\cftsubsubsecdotsep}{\cftnodots}
        \setlength{\cftsubsecindent}{.5em}
        \setlength{\cftsubsubsecindent}{1em}
        \addtocontents{toc}{\cftpagenumbersoff{subsection}}
        \addtocontents{toc}{\cftpagenumbersoff{subsubsection}}

    % Bibliography settings
        \bibliographystyle{abbrvnat}
    
    % Header settings
        \setlength{\headheight}{15pt}
    
    % Box settings
        \tcbuselibrary{skins, breakable}
    
    % Path to images
        \graphicspath{{images/}}
    
    % Code listing settings
        \lstset{language=java} 
        \usemintedstyle{friendly}
        \definecolor{bg}{rgb}{0.9,0.9,0.9}
    
    % Automata settings
        \usetikzlibrary{automata, positioning}
        \tikzset{
            shorten >= 1pt, node distance = 5em, on grid, auto, initial text =, initial distance = 4mm
        }
    
    % Sans-serif font settings
        \setsansfont[
            Ligatures=TeX,
            Extension=.otf,
            UprightFont=*-regular,
            BoldFont=*-bold,
            ItalicFont=*-italic,
            BoldItalicFont=*-bolditalic,
            Scale=0.9
        ]{texgyreadventor}
    
    % Header and footer settings
        % Contents page style
        \fancypagestyle{toc}
        {
            \fancyhf{}
            \lhead{\footnotesize\sffamily\projecttitle}
            \rhead{\footnotesize\sffamily\projectdate}
            \lfoot{\footnotesize\sffamily%
                \newcounter{footer_ct}\forloop{footer_ct}{0}{\value{footer_ct} < \value{authorCount}}{%
                    \getAuthorID{\arabic{footer_ct}}%
                    \ifnumless{\value{footer_ct}}{\value{authorCount}-1}{, }{}%
                }
            }
            \rfoot{\footnotesize\sffamily Contents}
            \renewcommand{\footrulewidth}{0.5pt}
            \setcounter{page}{0}
        }
        \pagestyle{fancy}
            \fancyhf{}
            \lhead{\footnotesize\sffamily\projecttitle~ --~~~~ \nouppercase{\rightmark}}
            \rhead{\footnotesize\sffamily\projectdate}
            \lfoot{\footnotesize\sffamily%
                \setcounter{footer_ct}{0}\forloop{footer_ct}{0}{\value{footer_ct} < \value{authorCount}}{%
                    \getAuthorID{\arabic{footer_ct}}%
                    \ifnumless{\value{footer_ct}}{\value{authorCount}-1}{, }{}%
                }
            }
            \rfoot{\footnotesize\sffamily Page \thepage~of \pageref{LastPage}}
            \renewcommand{\footrulewidth}{0.5pt}
            \renewcommand\sectionmark[1]{%
            \markright{\thesection\ #1}}
    
    % Section title settings
        \sectionfont{\raggedright\LARGE\normalfont\sffamily}
        \subsectionfont{\large\normalfont\sffamily}
        \subsubsectionfont{\it\sffamily}
        \renewcommand{\thesection}{\hspace{-1em}}
        \renewcommand{\thesubsection}{\hspace{-1em}}
        \renewcommand{\thesubsubsection}{\hspace{-1em}}
    
    % Box settings
        \tcbset{enhanced, skin = bicolor, breakable,
        colframe = dtured, coltitle = white, colupper = white, colback = dtured, colbacklower = white,
        fonttitle = \sffamily\bfseries\large, nobeforeafter,
        toptitle = 1mm, bottomtitle = 1mm, after title = \vspace{2mm}\hrule\vspace{-1mm}}
    
    % Table column settings
        \newcolumntype{L}[1]{>{\raggedright\let\newline\\\arraybackslash\hspace{0pt}}m{#1}}
        \newcolumntype{C}[1]{>{\centering\let\newline\\\arraybackslash\hspace{0pt}}m{#1}}
        \newcolumntype{R}[1]{>{\raggedleft\let\newline\\\arraybackslash\hspace{0pt}}m{#1}}
        
    % Array settings
        \pgfkeys{
            /authorname array/.is family, /authorname array, .unknown/.style = {\pgfkeyscurrentname/.initial = #1},
            /authorid array/.is family, /authorid array, .unknown/.style = {\pgfkeyscurrentname/.initial = #1}
        }

%%--------------%%
%% NEW COMMANDS %%
%%--------------%%

    % Vector command
        \renewcommand{\vec}[1]{
        \begin{bmatrix} #1 \end{bmatrix}}
    
    % Normal text with normal subscript index
        \newcommand{\ts}[1]{\ensuremath{_\textrm{#1}}}
        
    % Normal text with normal exponent
        \newcommand{\te}[1]{\ensuremath{^\textrm{#1}}}
        
% Author array
\newcounter{authorCount}
\newcommand\setAuthor[2]{\pgfkeys{/authorname array, \arabic{authorCount} = #1}%
\pgfkeys{/authorid array, \arabic{authorCount} = #2}\stepcounter{authorCount}}
\newcommand\getAuthorName[1]{\pgfkeysvalueof{/authorname array/#1}}
\newcommand\getAuthorID[1]{\pgfkeysvalueof{/authorid array/#1}}